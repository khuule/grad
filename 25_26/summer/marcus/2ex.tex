\documentclass[letter,12pt]{article}
\usepackage[utf8]{inputenc}
\input{/home/scout/univ/f21/preamble}
\pdfsuppresswarningpagegroup=1
\newcommand{\Author}{Kevin Le} 
\newcommand{\Title}{Homework x}

\author{\Author}
\title{\Title}
\titleformat{\subsection}[runin]
    {\normalfont\normalsize\bfseries}
    {Exercise \thesubsection.}{1em}{}
\titleformat{\subsubsection}[runin]
{\normalfont\normalsize}
{(\alph{subsubsection})}{1em}{}

\newcommand\subsectionbreak{\clearpage}
\renewcommand{\subsectionmark}[1]{\markright{Exercise  \thesubsection }}

\usepackage[margin=.75in,headheight=18pt,headsep=\baselineskip]{geometry}
\begin{document}

\pagestyle{fancy}
\fancyhead[L]{\Title}
\fancyhead[R]{\rightmark}
\setcounter{section}{2}
\subsection{}
\subsubsection{} Show that every number field of degree 2 over $\Q$ is one of the quadratic fields $\Q[\sqrt{m}],m\in \Z$.
\begin{proof}
    Suppose $K$ is a degree 2 number field. Every such field is of the form $\Q[\alpha]$ for an algebraic number $\alpha\in \C$. Since $K$ is degree 2, $\alpha$ must be the root of a degree 2 irreducible polynomial $f=ax^2+bx+c$. When considering the irreducibility of $f$ over $\Q$, we might as well consider a monic polynomial ($a=1$) after rescaling. Hence, $f$ is irreducible if and only if
    \[
    \frac{-b\pm \sqrt{b^2-4c}}{2}\not\in \Q,
    \]
    which happens if and only if $b^2-4c$ is squarefree. It's clear then that $\Q[\sqrt{b^2-4c}]=\Q[\alpha]$, since we have $\alpha=-\frac{b}{2}\pm \frac{1}{2}\sqrt{b^2-4c}\in \Q[\sqrt{b^2-4c}]$ and $\sqrt{b^2-4c}=\pm \frac{b}{2}\pm \alpha\in \Q[\alpha]$.
\end{proof}
\subsubsection{} Show that the fields $\Q[\sqrt{m}]$, $m$ squarefree, are pairwise distinct. (Hint: Consider the equation $\sqrt{m}=a+b\sqrt{n})$; use this to show that they are in fact pairwise non-isomorphic.
\begin{proof}
    Suppose $m$ and $n$ are squarefree with $m\ne n$. We want to show that $\Q[\sqrt{m}]$ and $\Q[\sqrt{n}]$ are pairwise distinct. Suppose they are equal, then $\sqrt{m}=a+b\sqrt{n}$ for some $a,b\in \Q$. Then, $m=a^2+b^2n+2ab\sqrt{n}$. That is, $2ab=0$ and $m=a^2+b^2n$. Then we must have $a=0$ or $b=0$, but if $b=0$, then $m=a^2$, contradicting the fact that $m$ is not a square of an integer, let alone a rational number. On the other hand, writing $b=\frac{x}{y}$ for $x,y\in \Z$, we obtain $y^2m=x^2n$. Since $m$ and $n$ are squarefree, this only happens if $y=x$ and $m=n$.
\end{proof}
\subsection{} Let $I$ be the ideal generated by 2 and $1+\sqrt{-3}$ in the ring $\Z[\sqrt{-3}]=\{a+b\sqrt{-3} : a,b\in \Z\}$. Show that $I\ne (2)$ but $I^2=2I$. Conclude that ideals in $\Z[\sqrt{-3}]$ do not factor uniquely into prime ideals. Show moreover that $I$ is the unique prime ideal containing $(2)$ and conclude that $(2)$ is not a product of prime ideals.
\begin{proof}
    Note that $I=(2)$ if and only if $1+\sqrt{-3}\in (2)$, but we clearly cannot have
    \[
        1+\sqrt{-3}=2(a+b\sqrt{-3})=2a+2b\sqrt{-3}
    \]
    for $a,b\in \Z$. Now, $I^2=(4,2+2\sqrt{-3},-2+2\sqrt{-3})$ while $2I=(4,2+2\sqrt{-3})$. Since $2+2\sqrt{-3}=4+(-2+2\sqrt{-3})$, a sum of the other two generators of $I^2$, we can conclude that $I^2=2I$. For the next claim, we want to show that $I$ and $(2)$ are prime ideals. We claim that 2 is prime. Suppose $2\mid (a+b\sqrt{-3})(c+d\sqrt{-3})=ac-3bd+(ad+bc)\sqrt{-3}$.
\end{proof}
\subsection{} Complete the proof of Corollary 2, Theorem 1.
\begin{proof}
We'll start with finishing the proof of Corollary 2. We have that for $\alpha=r+s\sqrt{m}$, $r,s\in \Q$, for $s\ne 0$, the monic irreducible polynomial over $\Q$ having $\alpha$ as a root is
\[
x^2-2rx+r^2-ms^2.
\]
Hence, $\alpha$ is an algebraic integer if and only if $2r$ and $r^2-m s^2$ are integers. Now if $m$ is squarefree, then we must have $m\equiv 1,2,3\gmod{4}$. 
\end{proof}
\subsection{}
\subsection{} Show that if $f$ is any polynomial over $\Z_p$ ($p$ a prime) then $f(x^p)=(f(x))^p$. (Suggestion: Use induction on the number of terms.)
\begin{proof}
    Recall that the binomial formula gives us for $f,g\in \Z_p[x]$, $(f+g)^p=f^p+g^p$. Induction gives us the result.
\end{proof}

\subsection{} Show that if $f$ and $g$ are polynomials over a field $K$ and $f^2\mid g$ in $K[x]$, then $f\mid g'$. (Hint: Write $g=f^2h$ and differentiate)
\begin{proof}
    If $f^2\mid g$ in $K[x]$, then we can write $g=f^2h$. By the product rule and chain rule for formal derivatives, we get $g'=2ff'h+f^2h'$ so that $g'=f(2f'h+fh')$. Hence, $f\mid g'$.
\end{proof}
\end{document}
